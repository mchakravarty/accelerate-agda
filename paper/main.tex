% 16pages; IFL 2012
%
\documentclass{llncs}
%
\usepackage{amssymb}
\usepackage{cite}
\usepackage{color}
% \usepackage[T1]{fontenc}

%\usepackage{amsmath}
%\usepackage{amsthm}
%\usepackage{proof}
\usepackage{alltt}

% UNICODE in Agda sources: more trouble than it is worth right now!

% \usepackage{bbm}
% \usepackage[greek,english]{babel}
% \usepackage{ucs}
% \usepackage[utf8x]{inputenc}
% %\usepackage{autofe}
% \DeclareUnicodeCharacter{8704}{\ensuremath{\forall}}
% % \DeclareUnicodeCharacter{8594}{\ensuremath{\mathbb{N}}}
% \DeclareUnicodeCharacter{8594}{\ensuremath{\to}}


\usepackage{fancyvrb}

\DefineVerbatimEnvironment{code}{Verbatim}{} % Add fancy options here if you like.
\DefineVerbatimEnvironment{hcode}{Verbatim}{formatcom=\color{blue},frame=single} % Add fancy options here if you like.


\begin{document}
%
\title{Agda Meets Accelerate}
% \subtitle{Extended Abstract}
\author{Peter Thiemann\inst{1} \and Manuel M. T. Chakravarty\inst{2}}
\institute{
  University of Freiburg, Germany,\\
  \email{thiemann@informatik.uni-freiburg.de}
\and
University of New South Wales, Sydney, Australia,\\
\email{chak@cse.unsw.edu.au}
}

\maketitle              % typeset the title of the contribution

\begin{abstract}  
  Embedded languages in Haskell benefit from a range of type extensions, such as type families, that are subsumed by dependent types. However, even with those type extensions, embedded languages for data parallel programming lack desirable static guarantees, such as static bounds checks in indexing and collective permutation operations.
  
  This observation raises the question whether an embedded language
  for data parallel programming would benefit from fully-fledged
  dependent types, such as those available in Agda. We explored that
  question by designing and implementing an Agda frontend to
  Accelerate, a Haskell-embedded language for data parallel programming aimed at GPUs. We discuss the potential of dependent types in this domain, describe some of the limitations that we encountered, and share some insights from our preliminary implementation.
\end{abstract}
\keywords{programming with dependent types, data parallelism}
\thispagestyle{plain}
\pagestyle{plain}
%
\section{Introduction}
\label{sec:introduction}

Generative approaches to programming parallel hardware promise to
combine high-level programming models with high performance. They are
particularly attractive for targeting restricted architectures that cannot efficiently execute
code aimed at conventional multicore CPUs. One prime example are GPUs
(graphics processor units), which require a
high degree of data parallelism, restricted control flow, and
custom tailored data access patterns to be efficient. Previous work
---for example, Accelerator~\cite{Tarditi:2006},
Copperhead~\cite{Catanzaro:EECS-2010-124}, and
Accelerate~\cite{ChakravartyKellerLeeMcdonellGrover2011}---
demonstrates that embedded array languages with a custom code
generator can meet those GPU constraints with carefully designed language constructs.

Given a host language with an expressive type system, it is
attractive to leverage that type system to express static properties
of the embedded language. For example, Accelerate, an embedded array
language for Haskell, uses Haskell's recent support for type-level
programming like GADTs and type families in that
manner~\cite{ChakravartyKellerLeeMcdonellGrover2011}. This design
choice is desirable for approaches relying on run-time code
generation:
each  potential fault at application run time should be discovered by
a compile-time fault in the embedded language. Moreover, static guarantees
hold the potential to improve the predictability of parallel
performance. 

Dependent types \cite{MartinLoef1984} are an established approach to certified programming,
where invariants are established in the form of types and proven at
compile time. Many of Haskell's type-level extensions used in
Accelerate approximate aspects of dependently-typed
programming. Hence, it is natural to ask whether fully-fledged
dependent types, such as those provided by Agda, improve the
specification of an embedded language like Accelerate, whether they
increase the scope of static guarantees, and whether they may be
leveraged to predict performance more accurately. 

This paper is a first investigation into this topic. It reports on a
partial port of Accelerate to a new, dependently-typed host language,
Agda \cite{Norell2008,BoveDybjerNorell2009}. Agda is particularly
suited to this port because of its foreign function interface to
Haskell, which enables it to directly invoke the functionality of
Accelerate. The main contributions of this paper are the following:
%
\begin{itemize}
\item We identify and discuss the challenges of combining generative embedded languages with dependent typing (Section~\ref{sec:dependent-types}).
\item We propose predicated arrays to overcome some of these challenges (Section~\ref{sec:limitations}).
\item We outline an implementation of the main parts of Accelerate in Agda using the Agda-Haskell FFI for code execution (Section~\ref{sec:implementation}).
\end{itemize}
%
Overall, our investigation has the following structure. After recalling some
background on Agda and Accelerate in Section~\ref{sec:background} and
describing related work in Section~\ref{sec:related-work}, 
Section~\ref{sec:dependent-types} discusses potential uses of
dependent types in an array-oriented data parallel language and how they were realized in our
implementation. Section~\ref{sec:limitations} considers conceptual 
problems and limitations that we ran into when constructing the Agda
frontend for Accelerate. Section~\ref{sec:implementation} explains
some technical details of the implementation and discusses some
example code. 

\noindent
Source code is available at \url{https://github.com/mchakravarty/accelerate-agda}.

\section{Background}
\label{sec:background}

\subsection{Agda}
\label{sec:agda}

Agda \cite{Norell2008,BoveDybjerNorell2009} is a dependently-typed
functional programming language. Its basis is a dependently-typed
lambda calculus extended with inductive data type families, dependent
records, and parameterized modules. At the same time, Agda is also a
proof assistant for interactively constructing proofs in an
intuitionistic type theory based on the work of Per Martin-L\"of
\cite{MartinLoef1984}. 

One attractive feature of Agda's inductive data type families is the
ability to construct indexed data types.
A familiar example for such an indexed data type is the type
\verb+Vec A n+ of vectors of fixed length \verb+n+ and elements of
type \verb+A+. This vector data types can be equipped with an access
operation that restricts the index to the 
actual length of the vector at compile time.\footnote{An
  identifier can be an almost arbitrary 
  string of Unicode characters except spaces, parentheses, and curly
  braces. Agda also supports mixfix syntax with the position of
  arguments indicated by underscores in the defining occurrence of an
  identifier.} 
\begin{verbatim}
data Nat : Set where
  zero : Nat
  suc  : Nat -> Nat

data Vec (A : Set) : Nat -> Set where
  []   : Vec A zero
  _::_ : {n : Nat} -> A -> Vec A n -> Vec A (suc n)
\end{verbatim}
The above defines the type \verb+Nat+ of natural numbers and an
indexed data type \verb+Vec A n+ where \texttt{A} is a type and
\texttt{n} is a natural number. The latter type comes with two 
constructors, \verb+[]+ for the vector of length zero and 
\verb+_::_+ for the infix cons operator that increases the length by one.

One way of writing a safe access operation first defines an indexed
type that encodes the required less-than relation on natural numbers.
\begin{verbatim}
data _<_ : Nat -> Nat -> Set where
  z<s : {n : Nat} -> zero < suc n
  s<s : {m n : Nat} -> m < n -> suc m < suc n
\end{verbatim}
Lines two and three of the definition encode named inference rules for
the cases that $0 < n+1$ (for all $n$) and that $m+1 < n+1$ if $m < n$
(for all $m,n$).

The access operation takes a vector of length \verb+n+, an index
\verb+m+, and a proof of \verb+m < n+ (a derivation tree) to produce
an element of the vector.  
\begin{verbatim}
get : {A : Set} {n : Nat} -> Vec A n -> (m : Nat) -> m < n -> A
get []        _       ()      -- impossible case
get (x :: xs) zero    z<s     = x
get (x :: xs) (suc m) (s<s p) = get xs m p
\end{verbatim}
This code cannot fail at run time because a caller has to
construct the proof tree for \verb+m < n+ before invoking
\verb+get+. Thus, an ``index out of bounds'' error cannot happen. 
(In Agda, arguments in curly braces are \emph{implicit
  arguments} that will be inferred if omitted in an application.)

\subsection{Accelerate}
\label{sec:accelerate}

Accelerate~\cite{ChakravartyKellerLeeMcdonellGrover2011} is a
\emph{generative} data-parallel array language embedded into Haskell, which targets
GPUs. Being generative, its data-parallel array
operations are not executed directly. Instead, Accelerate constructs abstract
syntax trees (AST) representing an entire data-parallel
subcomputation. These \emph{computation representations} are executed
using a \verb+run+ operation that accepts such a representation (of type
\verb+Acc a+), compiles it to GPU kernels, uploads it to a device,
executes it, and retrieves the results.\footnote{To distinguish
  Haskell code from Agda code, we display Haskell code in a blue box.}
%
\begin{hcode}
CUDA.run :: Arrays a => Acc a -> a
\end{hcode}
%
The type class constraint \verb+Arrays a+ restricts the result type to a single array or a tuple of arrays.

As computation representations of type \verb+Acc a+ are compiled at
application run time, all \verb+Acc+ compilation errors are effectively \emph{run-time errors} of the application. Hence, Accelerate uses a range of Haskell type system extensions to statically type Accelerate expressions, such that these run-time errors are avoided where possible. In particular, Accelerate uses GADTs\cite{PeytonJonesVytiniotisWeirichWashburn2006}, associated types
\cite{ChakravartyKellerJones2005}, and type families
\cite{SchrijversPeytonJonesChakravartySulzmann2008}. 

As a simple example of an Accelerate program, consider a function implementing a dot product:
%
\begin{hcode}
dotp :: Vector Float -> Vector Float -> Acc (Scalar Float)
dotp xs ys = let { xs' = use xs; ys' = use ys }
             in  fold (+) 0 (zipWith (*) xs' ys')
\end{hcode}
%
The types \verb+Vector+ and \verb+Scalar+ represent one- and zero-dimensional
arrays. Plain arrays, such as \verb+Vector Float+ are conventional Haskell arrays, using an unboxed representation to improve performance. However, when they are wrapped into the constructor \verb+Acc+, such as in \verb+Acc (Scalar Float)+, they represent arrays of the embedded language and are allocated in GPU memory, which in current high-performance GPUs is physically separate from CPU memory.

The \verb+use+ operation makes a Haskell array available in the
embedded language by wrapping it into the \verb+Acc+ constructor. It
amounts to copying it to GPU memory.\footnote{Accelerate employs
  caching to avoid the transfer of arrays that are already
  available in GPU memory.} The operations \texttt{fold} and
\texttt{zipWith} represent collective operations on Accelerate arrays,
effectively producing a representation of an array computation
yielding  a single float value (\texttt{Scalar Float}). The code relies on (type class) overloading: \texttt{0}, \texttt{(+)}, and
\texttt{(*)} are overloaded to construct abstract syntax. 

The types \verb+Scalar+ and \verb+Vector+ are type synonyms instantiating a shape-parameterised array type to the special case of zero and one dimensional arrays:
%
\begin{hcode}
type Scalar e = Array DIM0 e
type Vector e = Array DIM1 e
\end{hcode}

In the general type for \verb+use+, the class \verb+Elt+ characterizes all types that may be held in Accelerate arrays. These are currently primitive types and tuples.
%
\begin{hcode}
use :: Elt e => Array sh e -> Acc (Array sh e)
\end{hcode}
%
Common dimensions, such as \verb+DIM0+, \verb+DIM1+, and so on, are
predefined, but to enable shape polymorphic computations, along the
lines pioneered in the Haskell array library
Repa~\cite{keller-etal:repa}, shapes are inductively defined using
type-level snoc lists built from the data types \texttt{Z} and
\texttt{:.}. The use of snoc lists simplifies the type signatures of
fold operations that reduce or abstract over the least significant dimensions.
%
\begin{hcode}
data Z       = Z
data sh :. i = sh :. i
-- Types for often used dimensions
type DIM0 = Z
type DIM1 = DIM0 :. Int
-- and so on
\end{hcode}

\section{Related Work}
\label{sec:related-work}

Peebles formalizes parts of the Repa API % of the Haskell array library Repa
using Agda~\cite{peebles:derpa}. The formalisation relies on the same
shape structure as Accelerate, but array computations are neither
embedded nor can parallel high-performance code be generated. 

Swierstra and Altenkirch investigated the use of dependent types for
\emph{distributed} array
programming~\cite{swierstra-altenkirch:dep-types-for-distr-arrays,swierstra:more-dependent-types}.
Their notation for distributed arrays is inspired by the X10 language
\cite{Saraswat2009}. They focus on expressing locality awareness.

Dependent ML is an ML dialect with a restricted form of dependent types, which, among other applications, may be used to statically check array bounds~\cite{xi:dml-jfp}. However, only simple indexing and array updating are considered and not aggregate array operations, such as those provided by Accelerate.

Accelerator~\cite{Tarditi:2006} enables embedded GPU computations in C\# programs; it subsequently also added F\# support. However, no attempt is made to track properties of array programs statically. Similarly, Copperhead~\cite{Catanzaro:EECS-2010-124} embeds an array language into Python, but does not attempt to track information statically.

\section{Dependent Types for Accelerate}
\label{sec:dependent-types}

In this section, we investigate the potential uses of dependent typing
in a language like Accelerate and point out how they may be
implemented in Agda. First, we review some basics of the
embedding. 

\subsection{Embedding of Haskell Types}
\label{sec:embedding-types}

Accelerate supports a wide range of numeric types, characterized by
the type class \texttt{Elt}, as base types for
array computations. Almost all of these types lack a suitable
counterpart in Agda, which only supplies  computationally expensive
encodings for natural and rational numbers. For that reason, our embedding keeps the Haskell types 
abstract in Agda. To specify the types of functions that are polymorphic in such a Haskell type or depend on it
in some way, we have reified the possible element types as an Agda type
\texttt{Elt}:
\begin{verbatim}
data Elt : Set where
  Bool   : Elt
  Int    : Elt
  Float  : Elt
  Double : Elt
  Pair   : Elt -> Elt -> Elt
  -- and so on
\end{verbatim}
Corresponding to Haskell type classes that are used in Accelerate, our
embedding supplies predicates that characterize subsets. For example,
the set of numeric types is defined by a predicate
\texttt{Numeric}:\footnote{$\top$ is a one-element type,
  whereas $\bot$ is a type without elements. These types customarily
  represent truth and falsity.}
\begin{alltt}
Numeric : Elt -> Set
Numeric Int = \(\top\)
Numeric Float = \(\top\)
Numeric Double = \(\top\)
Numeric _ = \(\bot\)
\end{alltt}
The embedding declares further subsets all in the same style.

\subsection{Array Types}
\label{sec:array-types}

To demonstrate the Agda embedding in action, we translate the
dot product example from Section~\ref{sec:accelerate} to Agda.\footnote{In Agda,
  arguments in double curly braces are \emph{instance
    arguments}~\cite{DevriesePiessens2011} that are aggressively
  inferred. We use them like type class constraints in Haskell.}
\begin{code}
dotp : forall {E : Elt} {{p : Numeric E}} {n : Nat}
     -> PreVector n E -> PreVector n E -> Scalar E
dotp{E} xs ys = 
  let xs' = use xs
      ys' = use ys
  in  fold _+_ ("0" ::: E) (zipWith _*_ xs' ys')
\end{code}
Unlike the Accelerate code, this function is polymorphic with respect
to the array element type, provided it is numeric.
The length parameter $n$ ensures that the two input
vectors have the same size. The \texttt{PreVector} type of the
arguments corresponds to the plain \texttt{Vector} type in Accelerate,
whereas the result type \texttt{Scalar E} corresponds to \texttt{Acc
  (Scalar E)}---a piece of abstract syntax. 

The \texttt{use} function works as before, but its type
includes more information:
\begin{code}
use : {sh : Shape}{E : Elt} -> PreArray sh E -> Array sh E
\end{code}
Like \texttt{E}, the index \texttt{sh} is now an element of an
ordinary type instead of having to rely on type-level snoc
lists:\footnote{Recent work on Haskell's type system manages to avoid
  this issue \cite{YorgeyWeirichCretinJonesVytiniotisMagalhaes2012}.}
\begin{verbatim}
data Shape : Set where
  Z     : Shape
  _:<_> : Shape -> Nat -> Shape
\end{verbatim}
Asking for arrays of equal shape, as in the signature of \texttt{use},
means that the arrays have to have the exact same layout. The
\texttt{PreVector} and \texttt{Vector} types are just synonyms as in
Haskell:
\begin{verbatim}
PreVector n E = PreArray (Z :< n >) E
Vector n E    = Array    (Z :< n >) E
\end{verbatim}
The functions \texttt{fold}, \texttt{zipWith}, and \texttt{:::} are
discussed in the subsequent subsections. The functions \verb|_+_| and
\verb|_*_| both have the same type:
\begin{code}
_+_ _*_ : {E : Elt} {{p : Numeric E}} -> Exp E -> Exp E -> Exp E
\end{code}
They are restricted to arguments of numeric type and construct
abstract syntax for an addition or a multiplication by delegating to
the corresponding Accelerate functions. The type \texttt{Exp E}
denotes an AST of an expression of type \texttt{E}. 



\subsection{Exact Checking of Array Bounds}
\label{sec:exact-checking-array}

Accelerate's API features expressive type constraints that describe
the shape of the array arguments and results. These constraints ensure
that no shape mismatches occur (e.g., a 1D array cannot be
considered 2D), but they do not ensure at compile
time that the sizes of the dimensions match up. Such a mismatch
results in a run-time error.

As an example, consider the function \texttt{reshape}. 
It takes a target shape \texttt{sh} and an array of source shape
\texttt{sh'} and changes the layout of that array to \texttt{sh}. 
\begin{hcode}
reshape :: Exp sh -> Acc (Array sh' e) -> Acc (Array sh e)
\end{hcode}
For this reshaping to work correctly, the underlying
number of elements must remain the same. For example, while it makes sense
to reshape a two-dimensional $3\times 4$-array to a vector of size
$12$ or to a three-dimensional $3\times2\times2$-array, an attempt to
reshape to a $2\times5$-array should be rejected at
compile time.

As \texttt{Shape} is an ordinary data type in Agda, we can define a
\texttt{size} function that computes the number of elements stored in
an array of a certain shape.
\begin{verbatim}
size : Shape -> Nat
size Z = 1
size (sh :< n >) = size sh * n
\end{verbatim}
Now we can state an accurate type for \texttt{reshape} in Agda, which
involves an extra argument with a proof that the source and
target shapes have the same size.
\begin{alltt}
reshape : \{sh : Shape\} \{E : Elt\}
       -> (sh' : Shape) -> Array sh E -> (size sh \(\equiv\) size sh')
       -> Array sh' E
\end{alltt}
There is a subtle difference to the original signature. In
Accelerate, the first argument is an \emph{expression} that produces a
value of type \texttt{sh} at run time, whereas the Agda
\texttt{reshape} requires a \texttt{Shape} as its first argument.


Furthermore, functions like \texttt{map} and \texttt{zipWith} obtain
more precise types. The type of \texttt{map} tells us that the input
shape is identical to the output shape:
\begin{verbatim}
map : {A B} {sh} -> (Exp A -> Exp B) -> Array sh A -> Array sh B
\end{verbatim}
Similarly, the type of \texttt{zipWith} restricts its input arrays to
identical shapes:
\begin{verbatim}
zipWith : {A B C} {sh} -> (Exp A -> Exp B -> Exp C)
        -> Array sh A -> Array sh B -> Array sh C
\end{verbatim}
The latter type is more restrictive than the Accelerate implementation
of \texttt{zipWith}. Instead of checking the sizes of the input
arrays, it truncates them to the respective minima. We also developed
an Agda type that directly corresponds to this implementation. It
requires a binary function \texttt{isect} that computes the minimum of
two shapes of the same rank,
which we leave as an exercise to the reader.
\begin{alltt}
zipWith' : \{A B C\} \{shA shB\} \{p : rank shA \(\equiv\) rank shB\}
         -> (A -> B -> C)
         -> Array shA A -> Array shB B -> Array (isect shA shB p) C
\end{alltt}
\subsection{Associativity of Operations}
\label{sec:assoc-oper}

Some parallel reduction operations require their base operation to be
associative to return a predictable result. Here are two examples from
Accelerate. 
\begin{hcode}
fold  :: (Shape ix, Elt a) =>
         (Exp a -> Exp a -> Exp a) -> Exp a ->
         Acc (Array (ix :. Int) a) -> Acc (Array ix a)
fold1 :: (Shape ix, Elt a) =>
         (Exp a -> Exp a -> Exp a) ->
         Acc (Array (ix :. Int) a) -> Acc (Array ix a)
\end{hcode}
In both cases, the text of the documentation says that ``the first
argument needs to be associative'' and furthermore the \texttt{fold1}
documentation ``requires the reduced array to be non-empty''.
The second requirement can be enforced by asking for a suitable proof
object on each call of \texttt{fold1}:
\begin{verbatim}
fold1 : ... -> Array (sh :< n >) E -> (size sh * n > 0)
            -> Array sh E
\end{verbatim}
The first requirement can be rephrased to saying that the first two
parameters of \texttt{fold} together form a monoid, which requires an
associative operation with a unit element. The concept of a monoid
can be formalized in Agda, which has indeed been done in the standard
library. Unfortunately, the formalization from the library cannot be
used because Accelerate deals with ASTs, not with values. So, a
formalization is required that states that the meaning of an
AST-encoded function is associative and the meaning of another
AST-encoded constant is its unit element. Given that Accelerate
encodes AST construction using higher-order abstract syntax, such a
formalization is not straightforward. Moreover, even given expressions
with a fixed meaning, associativity has to be proved on a case by case
basis.

In any case, providing such information would be done by including an
additional argument that holds a suitable proof object, as in
\begin{verbatim}
fold : forall {E}{sh}{n}
     -> (f : Exp E -> Exp E -> Exp E) -> (e : Exp E)
     -> Array (sh :< n >) E -> IsMonoid f e -> Array sh E
\end{verbatim}
where
\begin{verbatim}
IsMonoid : forall {E} -> (Exp E -> Exp E -> Exp E) -> Exp E -> Set
IsMonoid f e = ( IsAssociative f , IsUnit f e)
\end{verbatim}

Some readers may object that neither addition nor multiplication of
floating point numbers is associative
\cite{DBLP:journals/csur/Goldberg91}. However, for advanced
optimizations, the exploitation of algebraic laws is a necessity and
the involved degradation of precision or change of result is accepted
or accounted for in the error estimates. Moreover, there are
other operations, like min or max, that are commonly used with
\texttt{fold}-like operations, which are truly associative. Last, but
not least, the associativity declarations serve as important
documentation that passing an inherently non-associative function will
produce unpredictable, implementation-dependent results.

\subsection{Embedding of Constants}
\label{sec:embedding-constants}

Accelerate relies on Haskell's built-in support for the type classes
\texttt{Num} and \texttt{Fractional} to embed constants. The Haskell
compiler reads each integer literal as a value of type
\texttt{Integer}, which is a built-in type of arbitrary precision
integers. To this value, Haskell applies the function
\texttt{fromInteger} that converts to the type expected by the
context. Similarly, floating point constants are read as values of
type \texttt{Rational} (\texttt{Integer} fractions) and then converted
using \texttt{fromRational}. Accelerate provides instances of these
type classes that define \texttt{fromInteger} and
\texttt{fromRational} to produce suitable AST fragments.

Because of Agda's lack of support for overloaded numeric literals,
we embed numeric literals for integers and floating point numbers
using a string with an explicit type annotation that determines the
parsing of the string. Here are some example embeddings:
\begin{verbatim}
"3.1415926" ::: Float
"6.0221415E23" ::: Double
\end{verbatim}
Recall that \texttt{Float} and \texttt{Double} are not
types, but rather values of type \texttt{Elt}.
The \texttt{:::} operation is the workhorse of the embedding:
\begin{verbatim}
_:::_ : (s : String) -> (E : Elt) 
     -> {{nu : Numeric E}} -> {p : T (s parsesAs E)} -> Exp E
s ::: E = Ex (constantFromString (EltDict E) (ReadDict E) s)
\end{verbatim}
The arguments \texttt{s} and \texttt{E} are explicit, but the
remaining ones are inferred by Agda.
As mentioned before, the argument \texttt{nu} is an instance argument; it is automatically
filled-in with a suitably typed value in scope
\cite{DevriesePiessens2011}. As before, the predicate 
\texttt{Numeric} plays the role of a type class that characterizes
the numeric types.

The function \texttt{parsesAs} dispatches on its ``type'' argument and
parses the string to check whether it is an acceptable literal of the expected type. The
function \texttt{constantFromString} is imported from Accelerate.
It is an overloaded function that requires two type dictionaries,
which are computed from \texttt{E} using the functions \texttt{EltDict}
and \texttt{ReadDict}. 

This setup results in a flexible way of handling literals. It has
worked well in our examples.

\section{Limitations}
\label{sec:limitations}

In a number of places, Accelerate's generativity limits the
applicability of dependent typing. We already mentioned that the
formalization of associativity or of the concept of a monoid cannot be
verified in Agda because such properties have to be asserted for abstract syntax.

For a related problem, consider an implementation of the \texttt{filter}
operation that takes a predicate and a source array and returns an
array that only contains the elements of the source array fulfilling
the predicate.  First of all, filtering only makes sense for
one-dimensional arrays, that is, for vectors. To see the second catch,
let's try to write down a dependent type signature for \texttt{filter}.
\begin{verbatim}
filter : forall {n m : Nat}{E : Elt}
       -> Vector n E -> (Exp E -> Exp Bool) -> Vector m E
\end{verbatim}
The problem is that the size of the result
cannot be determined statically. In fact, the only thing that we could
prove from an implementation of \texttt{filter} is that
\texttt{m} must be less than or equal to \texttt{n}.
However, we cannot even prove this restriction on \texttt{m} for two
reasons. First, the code of \texttt{filter} is not available as even
the underlying Haskell library only deals with abstract
syntax. Second, even if the code were available, there is still a
problem due to staging. The Accelerate implementation computes the length of the result only when the generated GPU code is executed. The function \texttt{Exp E -> Exp Bool} maps abstract syntax to abstract syntax; it does not directly implement a Boolean predicate. Hence, we cannot use the predicate in the result type of \texttt{filter} to more accurately constrain the size of the resulting vector --- unless we include an evaluator for Accelerate expressions. 

We might contemplate employing an existential type like
\begin{verbatim}
exists Nat (\ m -> m <= n -> Vector m E)
\end{verbatim}
but it is not possible to build such an
existential package because the evidence \texttt{m} is not available
when the existential package has to be constructed.

However, an alternative encoding of arrays can be
used which is compatible with filtering of elements. The idea of this
encoding is to keep all elements but mark those which are no longer
present because they have been filtered out.
There are several ways of implementing this idea. The simplest
approach is to pair up each element with a boolean flag that indicates
its presence, which we call \emph{predicated arrays}:\footnote{Using an array of \texttt{Maybe}-typed elements is not an option
  because Accelerate does not support \texttt{Maybe} types as array elements.}
\begin{verbatim}
FVector : Nat -> Elt -> Set
FVector n E = Vector n (Pair Bool E)
\end{verbatim}

In this encoding, filtering is quite simple because the length of
the \texttt{FVector} does not change. Furthermore, filtering could be
extended to multi-dimensional arrays, although the result might
require careful interpretation.
\begin{verbatim}
filterF : forall {n : Nat}{E : Elt}
        -> (Exp E -> Exp Bool) -> FVector n E -> FVector n E
filterF {n}{E} pred vec = map g vec
  where g : Exp (Pair Bool E) -> Exp (Pair Bool E)
        g bx = pair ((fst bx) && p (snd bx)) x
\end{verbatim}
The mapping operation that applies a function to each element of an
array becomes more complicated because it either has to 
materialize a dummy result for each absent element in the argument
vector or apply the function to absent elements, too. This makes \texttt{filter} reminiscent of the \texttt{where} statement of the SIMD language C$^*$~\cite{rose-etal:c-star}.
\begin{verbatim}
mapF : forall {n : Nat}{E F : Elt}
     -> Exp F -> (Exp E -> Exp F) -> FVector n E -> FVector n F
mapF {n}{E}{F} defaultF f vec = map g vec
  where g : Exp (Pair Bool E) -> Exp (Pair Bool F)
        g bx = if (fst bx) then (pair (fst bx) (f (snd bx)))
                           else (pair (fst bx) defaultF)
\end{verbatim}
Some operations can get rid of absent
elements. A fold operation which reduces a filtered
vector with a monoid returns a single value. In Accelerate, such a value has type
\texttt{Scalar}, which is a synonym for an array of dimension $0$.
\begin{verbatim}
foldF : forall {n : Nat}{E : Elt}
      -> (Exp E -> Exp E -> Exp E) -> Exp E
      -> FVector n E -> Scalar E
foldF f e vec =
  fold f e (map (\ bx -> if (fst bx) then (snd bx) else e) vec)
\end{verbatim}
Operations like \texttt{fold1} and the scan operations
extend to this representation, but they
% retain a notion of absent elements and
cannot revert to a
non-filtered representation. 

In the end, such a representation may not even lead to reduced efficiency on a GPU. As long
as all computations take the same path, all processing elements work
in unison. As soon as there are different paths in the same
computation step, then some elements will be idle for part of the
computation step. So it would be most advantageous to organize work as
uniformly as possible by reorganizing the array so that the present
and the absent elements are grouped together. A segmented array might
be a suitable representation.


\section{Implementation}
\label{sec:implementation}

Ordinarily, Agda is an interactive tool for constructing proofs and
verified programs. Programs may be run, which amounts to normalizing
Agda expressions, but this process is not very efficient.

Alternatively, an interactively developed program may be compiled to
Haskell using the Alonzo compiler. This compiler supports a Haskell
foreign function interface (FFI), which enables Agda programs to invoke
Haskell functions. 

Using this interface amounts to declaring a typed identifier in Agda
and then binding the identifier to a suitably typed Haskell
function. As an example, consider the import of the \texttt{use}
function.
\begin{code}
postulate 
  useHs : {E : Set}
      -> HsEltDict E -> HsArray HsDIM1 E -> Acc (AccArray HsDIM1 E)
  {-# COMPILED useHs       (\ _ -> Accel.use) #-}
\end{code}
The first three lines introduce the typed identifier \texttt{useHs}
and the last line is a pragma for the Alonzo compiler that binds the
Agda identifier \texttt{useHs} to the Haskell expression on the right.
But wait, this type looks very unpleasant and quite different to the
one mentioned in Section~\ref{sec:array-types}. This difference arises
because the type translation of Alonzo is unable to cope with the
index type \texttt{Shape}. For that reason, the interface uses a
simplified array type and adapter functions are required, in the worst
case, both on the Agda side and on the Haskell side of the interface.

At the foreign function interface level, all arrays are considered as
one-dimensional arrays. Additional arguments are passed to encode the
shape information as far as it is needed. The Agda adapter provides
the encoding of this structure and the Haskell adapter decodes it again. 

We believe that these adaptations only have a minor performance impact
because (1) most functions just manipulate abstract syntax, so that
only AST construction is affected, and (2) internally, Accelerate
considers all arrays as one-dimensional so that operations like
\texttt{reshape} are no-ops at run time.

Here is the Agda adapter for \texttt{use}:
\begin{verbatim}
use : forall {sh : Shape}{E : Elt} -> PreArray sh E -> Array sh E
use {sh}{E} (PA y) = Ar (useHs (EltDict E) y)
\end{verbatim}
It makes use of two wrapper types. \texttt{PreArray} wraps a
one-dimensional Haskell array using the constant \texttt{HsDIM1} (the
\texttt{DIM1} type shown in Section~\ref{sec:accelerate} imported from
Haskell via FFI) and the function \texttt{EltType} (not shown), which
interprets a value of type \texttt{Elt} as a Haskell type. The
latter types are also imported via FFI.
\begin{code}
data PreArray (E : Elt) : Shape -> Set where
  PA : {sh : Shape} -> HsArray HsDIM1 (EltType E) -> PreArray sh E
\end{code}
The \texttt{Array} type wraps an AST reference for an Accelerate
array, where \texttt{Acc} and \texttt{AccArray} are types imported
from Haskell.
\begin{code}
data Array (E : Elt) : Shape -> Set where
  Ar : {sh : Shape} -> Acc (AccArray HsDIM1 (EltType E)) -> Array sh E
\end{code}
The \texttt{EltDict} function translates a value 
\texttt{(E : Elt)} into a Haskell expression that evaluates to a
dictionary for the Haskell type of \texttt{E} for the Haskell type
class \texttt{Elt}. Such a dictionary is passed, whenever the
corresponding Haskell function has type class constraints.
\begin{code}
EltDict : (E : Elt) -> HsEltDict (EltType E)
\end{code}

The Haskell side of the adapter has several purposes. First, it
materializes the type class dictionaries from the encoding that we
just discussed. Second, it reconstructs sufficient information about
the array shape so that the intended operation can execute. Here is
the code for \texttt{Accel.use}, where the module name \texttt{A}
is a shorthand for \texttt{Data.Array.Accelerate}.
\begin{hcode}
use :: EltDict e -> Array A.DIM1 e -> A.Acc (A.Array A.DIM1 e)
use EltDict (ARRAY ar) = (A.use ar)
\end{hcode}
It does not have to reconstruct any information except the type class
constraint. This constraint is materialized using the type
\texttt{EltDict} below.
\begin{hcode}
data EltDict e where
  EltDict :: (A.Elt e) => EltDict e
\end{hcode}
This datatype is built such that each value captures the \texttt{Elt}
dictionary of type \texttt{e}. It remains to build such values for all
types that we want to transport across the FFI. These are the values
used by the (Agda) \texttt{EltDict} function. Here are two examples.
\begin{hcode}
eltDictBool :: EltDict Bool
eltDictBool = EltDict

eltDictInt :: EltDict Int
eltDictInt = EltDict
\end{hcode}

As an example for a function that requires more work on both sides,
consider the \texttt{fold} operation. 
\begin{code}
fold : forall {E}{sh}{n}
     -> (Exp E -> Exp E -> Exp E)
     -> Exp E
     -> Array (sh :< n >) E
     -> Array sh E
fold {E}{sh}{n} f (Ex e) (Ar a) =
  Ar (foldHs (EltDict E) (toHsInt (size sh)) (toHsInt n)
             (unwrap2 f) e a)
\end{code}
As values of type \texttt{Exp} also need a wrapper type in Agda (it is
not possible to import type constructors via the FFI), there is some
unwrapping going on for the \texttt{e} and \texttt{f}
arguments. The implementation of \texttt{fold} just calls the
\texttt{foldHs} function and encodes the information about the shape
in two integer arguments. Here, \texttt{size sh} is the size of the
result and \texttt{n} is the size of the dimension that is folded. As
these values are initially available as Agda natural numbers, they need to
be converted to Haskell numbers using the function \texttt{toHsInt}.

The \texttt{foldHs} function is defined via the FFI.
\begin{code}
postulate
  foldHs : {A : Set}
          -> HsEltDict A
          -> HsInt
          -> HsInt
          -> (AccExp A -> AccExp A -> AccExp A)
          -> AccExp A
          -> Acc (AccArray HsDIM1 A)
          -> Acc (AccArray HsDIM1 A)
  {-# COMPILED foldHs      (\_ -> Accel.fold) #-}
\end{code}
The Haskell adapter reconstructs the \texttt{Elt} dictionary
as before, but it also needs to reshape the one-dimensional array
representation into a two-dimensional one for executing the fold
operation. The two size arguments are required for exactly this
reshape operation. With that insight, the code is straightforward.
\begin{hcode}
fold :: EltDict a
     -> Int -> Int
     -> (A.Exp a -> A.Exp a -> A.Exp a)
     -> A.Exp a
     -> A.Acc (A.Array A.DIM1 a)
     -> A.Acc (A.Array A.DIM1 a)
fold EltDict size2 size1 f e a =
     (A.reshape (A.lift (A.Z A.:. size2))
      (A.fold f e
       (A.reshape (A.lift (A.Z A.:. size2 A.:. size1)) a)))
\end{hcode}

Fortunately, the \texttt{fold} example is about as complicated as the
adapter code gets. There are also many cases where at least one side
of the adapter code is trivial. However, each case must be considered
separately. 

\section{Conclusion}
\label{sec:conclusion}

We have built an experimental Agda frontend for the Accelerate
language. The goal of this experiment was to explore potential uses of
dependently-typed programming for data-parallel languages. 

At the moment, the outcome of the experiment is mixed. It is
successful, because we have been able to construct Agda functions for
a representative sample of Accelerate's functionality.
However, there was less scope for encoding extra information in the
dependent types than we had hoped for. Exact matching of array bounds
works, but results in restrictions (like the problems with
\texttt{zipWith} and filtering) that were not anticipated.

Exploiting algebraic properties did not work out in the intended way, mainly
because it boils down to asserting that some AST denotes an
associative function. However, these assertions cannot be proven: the
proof would have to apply the semantics to the AST, but the AST is an
abstract type in our implementation. An AST representation in Agda
might give us a better handle at this problem. 

In some places, the Agda frontend is less dynamic than Accelerate. In
a number of places, Accelerate accepts a run-time value of type
\texttt{Exp sh} for a shape argument, where the Agda frontend requires
a value of type \texttt{Shape}. To address this problem, we would have
to include a \texttt{Shape}-indexed encoding of the \texttt{Shape}
type in the \texttt{Elt} type so that we can describe the type of
an expression whose value has a certain shape. 

Finally, the type translation of Agda's FFI has a number of shortcomings that
cause problems when transporting information between Agda
and Haskell. One part of the problem is, unfortunately, the rich type
structure of Accelerate's API which already encodes many useful
constraints. An alternative, untyped (or less-typed) interface to
Accelerate would make the adaptation to an Agda frontend simpler.

%
% ---- Bibliography ----
%
\bibliography{local,abbrevs,papers,books,collections,misc,theses}
\bibliographystyle{abbrv}

\end{document}

%%% Local Variables: 
%%% mode: latex
%%% TeX-master: t
%%% coding: utf-8
%%% End: 
