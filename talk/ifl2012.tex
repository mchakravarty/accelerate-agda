\documentclass{beamer}
% 20+5 minutes
\usepackage{amsmath}
\usepackage[latin1]{inputenc}
\usepackage{graphicx}
\usepackage{amsmath}
\usepackage{amssymb}

\usepackage{theme/beamerthemeALU}



\AtBeginSection[]
{
  \begin{frame}<beamer>
    \frametitle{Outline}
    \tableofcontents[currentsection,currentsubsection]
  \end{frame}
}


\title{Agda Meets Accelerate}
\date{1.9.2012}
\author{Peter Thiemann \and Manuel Chakravarty}
\institute{%
  University of Freiburg, Germany,\\
  \texttt{thiemann@informatik.uni-freiburg.de}
\and
University of New South Wales, Sydney, Australia,\\
\texttt{chak@cse.unsw.edu.au}
}
\begin{document}

% begin titlepage
% "`normal"' front page template:
%\setbeamertemplate{title page}[default][rounded=true]
\setbeamertemplate{navigation symbols}{}
\begin{frame}[plain,label=fp]
		\advance\textwidth by 3.2em\relax
		\begin{minipage}{\textwidth}\par%
				\maketitle
		\end{minipage}
		\hspace*{2.5em}%
\end{frame}
\makeatother 
% end titlepage
\setbeamertemplate{navigation symbols}[horizontal]

\section{Introduction}


\begin{frame}
  \frametitle{Introduction}
  \begin{itemize}
  \item Domain-specific embedded languages
  \item Dependent types
  \item D-Generative programming
  \item Data parallelism
  \end{itemize}
\end{frame}

\begin{frame}
  \frametitle{Overview}
  \begin{block}{Mission}
    Construct a dependently-typed, generative, data parallel programming language
  \end{block}
  \begin{block}<2->{Means}
    \dots\ by creating an Agda frontend for the Accelerate language.
  \end{block}
\end{frame}

\section{Accelerate}
\begin{frame}
  \frametitle{The Accelerate Language}
  \begin{itemize}
  \item Data-parallel array language
  \item Expressive embedding in Haskell
  \item Generates GPU code at run time
  \item See Chakravarty et al, DAMP2011
  \end{itemize}
\end{frame}

\begin{frame}[fragile]
  \frametitle{Accelerate Example}
  \framesubtitle{Dot Product}
\begin{verbatim}
dotp :: Vector Float -> Vector Float
     -> Acc (Scalar Float)
dotp xs ys = let xs' = use xs
                 ys' = use ys
             in  fold (+) 0 (zipWith (*) xs' ys')
\end{verbatim}
  \begin{itemize}
  \item \texttt{dotp xs ys} returns a representation of code that
    computes the dot product of \texttt{xs} and \texttt{ys}
  \item References to inputs via \texttt{use}
  \item Specification of operations via HOAS (viz.\ \texttt{(+)} and \texttt{(*)})
  \end{itemize}
\end{frame}

\begin{frame}[fragile]
  \frametitle{Running Accelerate Code}
  \begin{itemize}
  \item To execute \texttt{dotp}, Accelerate provides several
    implementations of a \texttt{run} function:
\begin{verbatim}
CUDA.run :: Arrays a => Acc a -> a
\end{verbatim}
  \item This implementation
    \begin{itemize}
    \item compiles the code to GPU kernels
    \item loads the \texttt{use}d arguments along with the code onto
      the GPU
    \item executes and
    \item retrieves the result.
    \end{itemize}
  \end{itemize}
\end{frame}

\begin{frame}[fragile]
  \frametitle{Types in Accelerate}
  \begin{itemize}
  \item \texttt{Vector} and \texttt{Scalar} are suitable
    instantiations of a parameterized array type:
\begin{verbatim}
type Scalar e = Array DIM0 e
type Vector e = Array DIM1 e
\end{verbatim}
  \item Dimensions in turn are abbreviations for certain array shapes:
\begin{verbatim}
type DIM0 = Z
type DIM1 = DIM0 :. Int
-- <and so on>
\end{verbatim}
  \item Shapes are built from two basic constructors
\begin{verbatim}
data Z       = Z
data sh :. i = sh :. i
\end{verbatim}
  \end{itemize}
\end{frame}

\section{Agda}
\begin{frame}
  \frametitle{Agda}
  \begin{itemize}
  \item A dependently-typed functional programming language
    \begin{itemize}
    \item Inductive datatype families with dependent pattern matching
    \item Dependent records and parameterized modules
    \item Universe polymorphism
    \item \dots
    \end{itemize}
  \item Interactive proof assistant based on intuitionistic type theory
  \end{itemize}
\end{frame}

\begin{frame}[fragile]
  \frametitle{Indexed Types}
  \begin{itemize}
  \item Standard example: vector type indexed by its size
\begin{verbatim}
data Vec (A : Set) : Nat -> Set where
  []   : Vec A zero
  _::_ : {n : Nat} -> A -> Vec A n -> Vec A (suc n)
\end{verbatim}
  \item<2-> Enables the construction of statically safe access operations
    for vectors
\begin{verbatim}
get : forall {A : Set} {n : Nat} ->
      Vec A n -> (m : Nat) -> m < n -> A
get []        _       ()      -- impossible case
get (x :: xs) zero    p       = x
get (x :: xs) (suc m) (s<s p) = get xs m p
\end{verbatim}
  \end{itemize}
\end{frame}

\section{Dependent Types for Accelerate}

\begin{frame}
  \frametitle{Dependent Types for Accelerate}
  \begin{block}{Expectations}
    \begin{enumerate}
    \item Augment array shapes with static size information
    \item Correctness guarantees for fold-like operations
    \item Express all Haskell examples with stronger guarantees
    \end{enumerate}
  \end{block}
  \begin{block}<2->{Results}
    \begin{enumerate}
    \item Bounds checking works out.
    \item Not satisfactory.
    \item Some workarounds required.
    \end{enumerate}
  \end{block}
\end{frame}

\begin{frame}[fragile]
  \frametitle{Example: Dot Product in Agda-Accelerate}
  \vspace{-\baselineskip}
\begin{verbatim}
dotp : forall {n : Nat}
     -> PreVector n Float -> PreVector n Float
     -> Scalar Float
dotp xs ys = 
  let xs' = use xs
      ys' = use ys
  in
  fold _+_ ("0" ::: Float) (zipWith _*_ xs' ys')
\end{verbatim}
  \begin{itemize}
  \item \texttt{PreVector n Float} : a vector of $n$ elements of
    type \texttt{Float}
  \item \texttt{Scalar Float} : \textbf{code} that evaluates to a scalar of type \texttt{Float}
  \end{itemize}
\end{frame}
\begin{frame}[fragile]
  \frametitle{The Array Type}
  \begin{itemize}
  \item The array type is indexed over a shape
  \item Shapes contain size information
\begin{verbatim}
data Shape : Set where
  Z     : Shape
  _:<_> : Shape -> Nat -> Shape
\end{verbatim}
  \item \texttt{Vector} and \texttt{Scalar} are, again,
    specializations of a shape-indexed array type
\begin{verbatim}
Vector n E    = Array    (Z :< n >) E
Scalar E      = Array    Z          E
\end{verbatim}
  \end{itemize}
\end{frame}
\begin{frame}[fragile]
  \frametitle{Bounds Checking: zipWith}
\begin{verbatim}
zipWith : forall {A B C} {sh}
        -> (Exp A -> Exp B -> Exp C)
        -> Array sh A -> Array sh B -> Array sh C
\end{verbatim}
  \begin{itemize}
  \item Ensures that only arrays of the exact same shape and size are
    zipped together
  \item More restrictive than Accelerate that only checks the
    dimension and then takes the minimum
  \item But that can also be encoded
  \end{itemize}
\end{frame}
\begin{frame}[fragile]
  \frametitle{Accelerate's zipWith Function in Agda}
\begin{verbatim}
zipWith' : forall {A B C} {sha shb} {p : EqShape sha shb}
        -> (Exp A -> Exp B -> Exp C)
        -> Array sha A -> Array shb B
        -> Array (shmin sha shb {p}) C
\end{verbatim}
where\scriptsize
\begin{verbatim}
data EqShape : Shape -> Shape -> Set where
  EqZ : EqShape Z Z
  EqS : {s1 s2 n1 n2} -> EqShape s1 s2 -> EqShape (s1 :< n1 >) (s2 :< n2 >)

shmin : (s1 : Shape) -> (s2 : Shape) -> { p : EqShape s1 s2 } -> Shape
shmin .Z .Z {EqZ} = Z
shmin .(s1 :< n1 >) .(s2 :< n2 >) {EqS {s1} {s2} {n1} {n2} y} =
    shmin s1 s2 {y} :< min n1 n2 >
\end{verbatim}
\end{frame}

\begin{frame}[fragile]
  \frametitle{Bounds Checking: fold and fold1}
  \vspace{-\baselineskip}
  \begin{itemize}
  \item \texttt{fold} and \texttt{fold1} take a binary operation and
    reduce one dimension of an array using the operation
\begin{verbatim}
fold op e []          = e
fold op e [a]         = a
fold op e [a1 ... an] = op (fold op e [a1 ... ai])
                           (fold op e [ai+1 ... an])
\end{verbatim}
  \item \texttt{fold} works for any array, but \texttt{fold1} elides
    the \texttt{e} argument and requires a non-empty array
  \item The latter restriction is not expressed in Haskell-Accelerate
  \end{itemize}
  \begin{block}<2->{\vspace{-\baselineskip}}
\vspace{-\baselineskip}\small
\begin{verbatim}
fold  :: (Shape ix, Elt a) =>
         (Exp a -> Exp a -> Exp a) -> Exp a ->
         Acc (Array (ix :. Int) a) -> Acc (Array ix a)
fold1 :: (Shape ix, Elt a) =>
         (Exp a -> Exp a -> Exp a) ->
         Acc (Array (ix :. Int) a) -> Acc (Array ix a)
\end{verbatim}
\vspace{-\baselineskip}
  \end{block}
\end{frame}

\begin{frame}[fragile]
  \frametitle{Agda Signatures for fold and fold1}
\begin{verbatim}
fold : forall {E}{sh}{n}
     -> (Exp E -> Exp E -> Exp E) -> Exp E
     -> Array (sh :< n >) E 
     -> Array sh E

fold1 : forall {E}{sh}{n}
      -> (Exp E -> Exp E -> Exp E)
      -> Array (sh :< suc n >) E
      -> Array sh E
\end{verbatim}
\end{frame}
\begin{frame}[fragile]
  \frametitle{Another Issue with fold and fold1}
  \begin{itemize}
  \item \texttt{fold op e} and \texttt{fold1 op}  partition their array
    argument and apply \texttt{op} in a bracketing that is a-priori unknown
  \item For deterministic results, \texttt{op} must be associative and
    (in case of \texttt{fold}) \texttt{e} should be a unit of \texttt{op}
  \item<2-> Such restrictions can be expressed in Agda
  \item<3-> Might ask for a signature like this
\begin{verbatim}
fold : ... -> Monoid E
     -> Array (sh :< n >) E -> Array sh E
\end{verbatim}
  \end{itemize}
\end{frame}
\begin{frame}[fragile]
  \frametitle{A Monoidal Catch}
An encoding of monoids in Agda:\scriptsize
\begin{verbatim}
record Monoid (E : Set) : Set where
  field
    op : E -> E -> E
    un : E
    op_assoc : forall {x y z} -> op (op x y) z = op x (op y z)
    op_un_le : forall {x} -> op un x = x
    op_un_re : forall {x} -> op x un = x
\end{verbatim}
\normalsize
\begin{block}<2->{The Catch}
  In the signature of \texttt{fold} \dots
  \begin{itemize}
  \item \texttt{E} is \textbf{not} a \texttt{Set}, but an element of a
    data type that encodes Haskell datatypes
  \item the type of the operation is
    \texttt{(Exp E -> Exp E -> Exp E)}\\
    a HOAS encoding of the operation's \textbf{syntax}
  \item<3-> Associativity of syntax is meaningless \dots 
  \end{itemize}
\end{block}
\end{frame}

\section{Limitation}

\begin{frame}[fragile]
  \frametitle{A Limitation}
  \begin{itemize}
  \item Haskell-Accelerate admits implementing a \texttt{filter}
    operation (for vectors) with the usual meaning
  \item<2-> Surprisingly, it cannot easily be implemented in Agda-Accelerate!
  \item<2-> Why is that?
  \end{itemize}
  \begin{block}<3->{Implementation attempt}
\begin{verbatim}
filter : forall {n m : Nat}{E : Element}
       -> Vector n E -> (Exp E -> Exp Bool) 
       -> Vector m E
\end{verbatim}
    \begin{itemize}
    \item Problem: What is \texttt{m}?
    \end{itemize}
  \end{block}
\end{frame}

\begin{frame}[fragile]
  \frametitle{A Workaround}
  \begin{itemize}
  \item Provide a special type of filtered vectors that pair each
    element with a boolean indicating its presence
\begin{verbatim}
FVector : Nat -> Element -> Set
FVector n E = Vector n (Pair Bool E)
\end{verbatim}
  \item<2-> Now filtering becomes straightforward 
\begin{verbatim}
filterF : forall {n : Nat}{E : Element}
        -> (Exp E -> Exp Bool)
        -> FVector n E -> FVector n E
filterF {n}{E} pred vec =
  map g vec
  where
  g : Exp (Pair Bool E) -> Exp (Pair Bool E)
  g bx = pair ((fst bx) && p (snd bx)) x
\end{verbatim}
  \end{itemize}
\end{frame}
\begin{frame}[fragile]
  \frametitle{More Operations on Filtered Vectors}
  \vspace{-\baselineskip}
  \begin{block}{Mapping}
  \vspace{-\baselineskip}\small
\begin{verbatim}
mapF : forall {n : Nat}{E F : Element}
     -> Exp F -> (Exp E -> Exp F) 
     -> FVector n E -> FVector n F
\end{verbatim}
    requires a dummy element of \texttt{F} to put in the result when
    the corresponding source entry is not present
  \end{block}
  \begin{block}<2->{Folding}
  \vspace{-\baselineskip}\small
\begin{verbatim}
foldF : forall {n : Nat}{E : Element}
      -> (Exp E -> Exp E -> Exp E) -> Exp E
      -> FVector n E -> Scalar E
foldF f e vec =
  fold f e (map (\ bx -> if (fst bx) then (snd bx) else e) vec)
\end{verbatim}
    gets out of the filtered vector back to a ``real'' value
  \end{block}
\end{frame}

\section{Wrap Up}

\begin{frame}
  \frametitle{In the paper}
  \begin{itemize}
  \item More information on the representation of types and constants
  \item Further examples
  \item Description of the implementation using the foreign function
    interface of the Alonzo compiler
  \end{itemize}
\end{frame}

\begin{frame}
  \frametitle{Conclusion}
  \begin{itemize}
  \item Only partially successful
  \item Bounds checking works well
  \item Large part could be done using lifted datatypes
  \item Generative languages do not fit well with dependent types
  \end{itemize}
\end{frame}

\begin{frame}[fragile]
  \frametitle{Example: Polymorphic Dot Product in Agda-Accelerate}
  \vspace{-\baselineskip}
\begin{verbatim}
dotp : forall {E : Element} {{p : IsNumeric E}} {n : Nat}
     -> PreVector n E -> PreVector n E -> Scalar E
dotp{E} xs ys = 
  let xs' = use xs
      ys' = use ys
  in
  fold _+_ ("0" ::: E) (zipWith _*_ xs' ys')
\end{verbatim}
  \begin{itemize}
  \item \texttt{PreVector n E} : a vector of $n$ elements of
    type \texttt{E}
  \item \texttt{Scalar E} : code that evaluates to a scalar of type \texttt{E}
  \item \texttt{IsNumeric E} restricts \texttt{E} to numeric types
  \end{itemize}
\end{frame}

\end{document}
